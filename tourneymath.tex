\documentstyle{report}

% \pagestyle{empty}

\setlength{\parindent}{0pt}
\setlength{\parskip}{2.5mm}

%\setlength{\oddsidemargin}{0.5in}
%\setlength{\evensidemargin}{0.5in}
%\setlength{\textwidth}{8.5in}

%\setlength{\topmargin}{0.0in}      % above Header
%\setlength{\headheight}{0.0in}     % Header
%\setlength{\headsep}{0.0in}        % between header and body
%\setlength{\topskip}{0.0in}        % skip at top to first line of text
%\setlength{\textheight}{11.0in}
%\setlength{\fcootskip}{0.0in}      % size for Footer

\begin{document}


%\chapter{Mathematics, Combinatorics and Scheduling of a drop down  tournament}
\begin{center}
{\it Peter Teuben - draft 28-oct-99}
\end{center}

\section*{Tournament Combinatorics and Scheduling}



\section*{Basic Combinatorics}

Let us start with a single drop-down draw of $N$ 
players\footnote{for sake of simplicity, number of players and
courts will be  powers of 2, typically 4, 8, 16, 32, 64}. 
We will consider three common drop-down formats:

\begin{enumerate}
\item
Single elimination. If you loose, you are out of the draw. 
Each player will only be guaranteed 1 match in this format.
\newline
This format is also known as ``{\bf A}''.

\item
Double elimination: two draws, A and B, 
where the B contains all first round loosers from the main (A) bracket.
Each player will be guaranteed a minimum of 2 matches.
If you loose after that, you can go home.
\newline
This format is also known as ``{\bf A/B}''.

\item
Triple elimination: four draws, A, B, C and D. The C draw 
contains all first
round loosers in A, the B contains all second round loosers in A, and the 
D contains all first round loosers in C. 
Each player will be guaranteed a minimum of 3 matches.
If you loose after that, you are done for the day.
\newline
This format is known as ``{\bf A/B/C/D}''.
\end{enumerate}

Now, for these three formats, let us compute some basic numbers
like the number of matches played, the number of rounds needed
to finish the draw, etc. They are summarized in table~\ref{basic}

\newpage

\begin{table}[h]
\begin{center}
\begin{tabular}{|r||r|r|r|}
\hline
format:           &       A                 &         A/B     &   A/B/C/D \\
                   & Single                 &  Double          & Triple  \\
\hline

entries in the draw & $2N-1$            &      $ 3N-2$        &  $ 4N-4 $    \\

&&&\\

total number of matches        & $N-1$                &    ${ 3N/2}$ - 2 &  $2N-4 $ \\

&&&\\

% first round matches & $N \over 2$             &  $N \over 2$ & $N \over 2$ \\
first round matches & $N/2$             &  $N/2$ & $N/2$ \\

&&&\\

time slices for scheduling  & $\log_2(N)$  & 
                $\log_2{( {N/2} )} + 1$  & 
                $\log_2{( {N/4} )} + 2$  \\

&&&\\

\hline

\end{tabular}
\caption{Basic numbers for N players in a single event draw 
in three different drop-down formats}
\label{basic}
\end{center}
\end{table}

Now let us extend this to compute how many matches are needed for
a full tournament. This time we assume that for given $N$ players,
half of them ($N/2$) will be male, and half female. Also, they all
nice pair up in doubles and mixed events. So, there will be
$N/2$ players in both singles (MS and WS), 
$N/4$ teams in both doubles (MD, WD), and 
$N/2$ teams in mixed (XD). Using the numbers computed in table~\ref{basic}
we then find the following results, as summarized in 
table~\ref{full}

\begin{table}[h]
\begin{center}
\begin{tabular}{|r|rrr|}
\hline
        &   A  &    A/B   &   A/B/C/D  \\
        & Single  &  Double   & Triple  \\
\hline

Players &&&\\
  N     &  2N-5      &  3N-10  &      4N-20    \\
4       &   3       &   -       &        -      \\
8      &    11      &   14      &       -      \\
16      &   27      &   38     &        44      \\
32      &   59      &   86     &        108      \\
64      &   123     &   182     &       236      \\
128      &  251     &   374     &       492      \\
&&&\\
Courts &&&\\
         &   N/2    &  N/2        &     N/2  \\

\hline

\end{tabular}
\caption{Basic numbers for a full
5-event (MS,WS,MD,WD,XD) tournament with N/2 men and N/2
women, filled draws, 3 events per player.}
\label{full}
\end{center}
\end{table}

\newpage

\section{Scheduling on limited courts}


The above derivations all assume that there are plenty of courts to 
put all matches on court when they can be played. Of course in reality
this is not the case. For a simple draw of $N$ players, one needs
$N/2$ courts. So, let us assume there are only $C$ courts
\footnote{again we are assuming $C$ is a power of 2},
where $C$ will often be less than $N/2$. 
The number of timeslices (rounds of full courts if you wish)
needed to finish a single elimination A draw
of $N$ players can be shown to be:
$$
	\log_2{N} + \sum_{i=1}^{ {N} \over {4C} }{ (2^i - 1)}
	\eqno{N \ge 4C}
$$
where the last summation term is only needed when the number of courts
is not sufficient to place all first rounds on court at the same time, 
{\it i.e.} ($N \ge 4C$). Using the well known formula 
$$
	\sum_{k=0}^{n} x^k = {  {x^{n+1} - 1} \over {x-1} }
$$
we then obtain
$$
	\log_2{N} + 2^{ {{N} \over {4C}} + 1} +  { {N} \over {4C} }  -  2
	\eqno{N \ge 4C}
$$

%  { {N} \over {4C} }

which for a few common cases are computed in Table~\ref{timing} below.

\begin{table}[h]
\begin{center}
\begin{tabular}{|r|rrrrrrr|}

\hline
~~~~$N$   &   2   &   4   &   8   &   16  &   32 &    64  &  128 \\

C~~~~~~ & & & & & & & \\

\hline

1       &   1   &   3   &   7   &   15  &   31  &   63  &   127  \\
2       &   1   &   2   &   4   &   8   &   16  &   32   &  64   \\
4       &   1   &   2   &   3   &   5   &   9   &   17   & 33    \\
8       &   1   &   2   &   3   &   4   &   6   &   10   & 18   \\
16      &   1   &   2   &   3   &   4   &   5   &   7  &  11    \\
32      &   1   &   2   &   3   &   4   &   5   &   6  &  8   \\
\hline
\end{tabular}
\caption{Scheduling: number of timeslices needed for given
number of players (N) and number of courts (C)}
\end{center}
\label{timing}
\end{table}

We can expand this formula for a double elimination A/B,

$$
        2\log_2{N/2} + 2^{ {{N} \over {8C}} + 2} +  { {3N} \over {4C} }  -  4
	\eqno{N \ge 8C}
$$

and triple elimination A/B/C/D format,

$$
        4\log_2{N/4} + 2^{ {{N} \over {16C}} + 3} +  { {3N} \over {4C} }  -  8
	\eqno{N \ge 16C}
$$





\end{document}
